
\begin{figure}[ht]
  \begin{subfigure}{0.3\textwidth}
    \resizebox{\linewidth}{!}{
      \begin{tikzpicture}
        \begin{feynman}
          \vertex (a) {\( q \)};
          \vertex[below right=of a] (b);
          \vertex[below left=of b] (c) {\( q \)};
          \vertex[right=of b] (d) {\( g \)};
          \diagram*{
            (a) -- [fermion] (b),
            (b) -- [fermion] (c),
            (b) -- [gluon] (d),
          };
        \end{feynman}
      \end{tikzpicture}
    }
  \end{subfigure}
  \begin{subfigure}{0.3\textwidth}
    \resizebox{\linewidth}{!}{
      \begin{tikzpicture}
        \begin{feynman}
          \vertex (a) {\( g \)};
          \vertex[below right=of a] (b) ;
          \vertex[below left=of b] (c) {\( g \)};
          \vertex[right=of b] (d) {\( g \)};
          \diagram*{
            (a) -- [gluon] (b),
            (b) -- [gluon] (c),
            (b) -- [gluon] (d),
          };
        \end{feynman}
      \end{tikzpicture}
    }
  \end{subfigure}
  \begin{subfigure}{0.3\textwidth}
    \resizebox{\linewidth}{!}{
      \begin{tikzpicture}
      \begin{feynman}
        \vertex (a) {\( g \)};
        \vertex[below right=of a] (b) ;
        \vertex[below left=of b] (c) {\( g \)};
        \vertex[above right=of b] (d) {\( g \)};
        \vertex[below right=of b] (e) {\( g \)};
        \diagram*{
          (a) -- [gluon] (b),
          (b) -- [gluon] (c),
          (b) -- [gluon] (d),
          (b) -- [gluon] (e),
        };
      \end{feynman}
    \end{tikzpicture}
    }
  \end{subfigure}
  \caption{Feynman diagrams for the fundamental interaction vertices of QCD.}
  \label{fig:feynman-diagrams}
\end{figure}
